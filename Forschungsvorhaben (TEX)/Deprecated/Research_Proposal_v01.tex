As a student of business information systems at the Technische Universität München (TUM), it is very common to learn and do modeling of software and business processes. However, the typical scenario used in these exercises might often diverge from the ones in real companies. At least in terms of complexity and size. \\
The motivation for this thesis originates from these exercises and the urge of using these techniques and modeling languages in a real world scenario. Even more compelling would be the chance to use them in a real company or start-up. \\
Fortunately the opportunity of working for a start-up, eKulturPortal, and using both learned and new modeling languages such as Case Management Model and Notation (CMMN), Business Process Model and Notation 2.0 (BPMN) and Decision Modeling and Notation (DMN) was provided by agon and TUM Chair for Application and Middleware Systems, represented by Martin Jergler. \\
Agon is small business whose core business area is producing plays. However, as the business is pretty small employing only five core employees and several freelancer or contracted employees, lots of the work surrounding producing plays has to be done manually. Their goal is to develop a browser-based tool in order to automate on the one hand these tasks, and on the other hand providing a meta software that is capable of adapting the process engine to the company's processes. \\
As we now introduced the theoretical part and the working context, it is important to define the way this thesis contributes to both practical parts and theoretical parts which leads us to the development of the research questions. \\
Several paragraphs above we mentioned using CMMN as a new modeling standard for case management. CMMN is a product by the Object Management Group (OMG) and embedded in their modeling language family. Two popular members of that family are the Unified Modeling Language (UML) and the Business Process Model and Notation language (BPMN). Both are well known and a lot of research has been done in the past to figure out best practices and develop synergies for them. Now CMMN is a relatively new standard since it was introduced by OMG in 2014 \cite{cmmnSpec}. \\
The eKulturPortal project ,management is interested in using multiple techniques to model their complexity and, as a long-distance goal, use a process engine in the background to let the user adapt the software to their own company's processes. This is possible by using either BPMN business rules or the Decision Model and Notation specification by OMG. \\
In this thesis we want to do research both on using the new DMN and CMMN specifications by OMG and on using techniques to discover or mine processes correctly for the eKultur platform. Initially the methodology for Process Discovery will be evaluated. 