
\documentclass[11pt,a4paper,draft]{article}
\usepackage{authblk}
\usepackage[square,sort]{natbib}
\usepackage[utf8]{inputenc}
\usepackage[english]{babel}
\usepackage{amsmath}
\usepackage{amsfonts}
\usepackage{amssymb}
\usepackage[left=2cm,right=2cm,top=2cm,bottom=2cm]{geometry}
\title{Research Proposal \\ \large Version 1.0}
\author{Dominik Gerbershagen}
\affil{Chair for Application and Middleware Systems\\ Department of Informatics\\ Technische Universität München}

\date{\today}
\begin{document}
\maketitle
\section{Introduction}
Developing software from scratch can be done in several ways. One is the model-driven approach using various software, process and data models. In terms of process models, many different modeling languages exist, each with its own syntax and semantics. In the previous years the Business Process Model and Notation language by the Object Management Group (OMG) has become a quasi-standard for modeling business processes. 
BPMN, however, has downsizes which will be discussed in this thesis. One of them is the application of business rules. 
A business rule can vary from a discount schedule  a product to a complex decision-tree. To apply a business rule in BPMN it is necessary to either add a .csv document, such as an Excel file, or to implement the decision making in the actual code, depending on the tool the modeler is using. The code, however, is not part of the model anymore. Consequently the separation of domain experts, process analysts and programmers, as \citep[see chapter 5]{Dumas2013} describes, dilutes. A process analyst needs to be able to write the business rule as a java class or as a .csv file, which is worse to understand for the domain expert than the actual model. Furthermore a BPMN model implementing decisions using gateways can become unclear and hard to understand. 
A new way of separating the decision model from the business process model while not necessarily using .csv files or java classes serves the Decision Model and Notation (DMN) standard proposed by the Object Management Group. In the first part, this thesis investigates the potential of DMN and evaluates the ability to link both, the DMN and BPMN languages. A special focus lies on the comparison between modeling business rules with the common BPMN technique and the new DMN specification. As a result, the two graphs will be evaluated in concerning the information flow, the comprehensibility and the compatibility to other modeling languages. 

The second part investigates another new specification by OMG: the \textit{Case Management Model and Notation}. 
BPMN is and has always been a modeling language for routine processes \citep[see]{Zeising_2014}. The common understanding of processes is the linear execution of tasks, done by sometimes one ore more departments of a company. At the end of every process either a new value is created or the value of a product or a service is incremented. This workflow is appropriate for companies creating something. To be more abstract: BPMN works good with a value chain in accordance to Michael E. Porter, who developed it in the 1980s. 
Thirty years later, the business environment has diversified from a value chain driven model to a service oriented one. This leads to less strict processes and more flexible ones "(...) depending on evolving circumstances and ad hoc decisions by knowledge workers regarding a particular decision (...)" \citep{cmmnSpec}. BPMN does not provide the opportunity to model such flexible ad hoc processes. This raises the question if BPMN and CMMN work together designing processes which can be on the one side well structured but also leave enough room for ad hoc decisions. In which case is the usage of CMMN more appropriate than BPMN or even DMN? How do those three specifications work together in a (software) project? Evaluating the possibilities and disadvantages of CMMN is the second part of this thesis. \\
Summarizing the paragraph above leads to three research questions: 
\begin{itemize}
\item Investigation of the new Decision Model and Notation specification published by the Object Management Group, extracting the downsides and advantages especially concerning the vast modeling of gateways in BPMN. Is DMN the solution to simplify decision-modeling? 
\item Investigation of the Case Management Model and Notation specification by the Object Management Group, particularly how case modeling can be applied in a model-driven software development project. 
\item How do CMMN, DMN and BPMN work together in a model-driven software project? Is there a valid possibility to combine all three specifications in one model? Is it possible to improve the process and information flow, readability and eventually implementation of the model by the developer?   
\end{itemize}

\section{Related work and background information}
\subsection{CMMN}
Case Management and knowledge work are not brand new inventions that have been created in the past few years. "Peter F. Drucker made the first reference to knowledge work in (...) 1959 (...)" \citep{Motahari-NezhadSwenson2013}. A current "overview and research challenges" provide \citep{Motahari-NezhadSwenson2013} who explain the difference between business process management and adaptive case management. They briefly sum up the state of the art in case management technology and the next generation solutions. 
Mentioning technology and tools for Case Management, CMMN and Adaptive Case Management, there are many articles dealing with these topics. \citep{OsuszekStanek2015} describe how adaptive case management can be implemented in businesses and integrated in Enterprise Resource Planing systems (ERP). Additionally they approach a new architecture which decouples decision logic, knowledge work and process flows. All this leads to a better handling of information and an optimization of business modeling. 
Another practical example provide \citep{Kuzin2013} explaining the company's approach towards an implementation of the CMMN paradigm. This includes the ability to change requirements or orders during run-time, which is one of the major aspects in their system. To achieve this goal, they first set up a meta model of their order-based system and enhanced it afterwards. 
These practical examples are important in order to evaluate the compatibility with the CMMN specification and other modeling languages, specifically BPMN. They also provide a good overview of how to combine modeling techniques and how they are realized as a system in companies. 
A more theoretical approach to case management and CMMN particularly provide \citep{WangTraore2014} and \citep{Zeising_2014}. They both do research on transforming CMMN into different languages. \citep{WangTraore2014} do model-to-model transformation from CMMN to  DDML (DEVS-driven Modeling Language) which is used to formalize CMMN and analyze it afterwards. \citep{Zeising_2014} have a similar approach, but a different goal. Due to weaknesses of CMMN, the language cannot be used to create a platform for both agile and route processes. They describe agile processes as the ones "(...) of which the exact flow cannot be determined completely a priori" \citep{Zeising_2014}, which is a fundamental characteristic of knowledge work and the reason why case management is so important for many industries. Coping with CMMN's downsides they build their platform on a "rule-based cross-perspective and model intermediate language on textual basis, (...) called \textit{Declarative Process Intermediate Language (DPIL)}" \citep{Zeising_2014}. \\
A useful source for evaluating CMMN as a standardization for adaptive case management is \citep{KurzSchmidtFleischmannEtAl2015}. The subtitle \textit{Examining the applicability of a new OMG standard for adaptive case management} is a good foundation to see how OMG met the expectations from the industry and researchers. This paper sets up requirements deriving from different sources described in detail in section two \citep[see][section 2]{KurzSchmidtFleischmannEtAl2015}. At the end of their paper, they evaluate how good the requirements were fulfilled by the CMMN standardization and provide feedback for future improvements. 

\subsection{DMN}
The Decision Model and Notation standardization was meant to improve the \textit{separation of concerns} \citep{BiardMauffBigandEtAl2015} which is the decoupling of decision logic and the control-flow. Biard et al. investigate how the new standard DMN can be used for decoupling BPMN and the decisions modeled as gateways. Decision-modeling is not typically included in control-flow oriented modeling languages. BPMN has not the power to model vast decision-trees due to the gateway restrictions. \citep{BatoulisMeyerBazhenovaEtAl2015} even calls it a "(...) [misuse] for modeling decision logic". They found an autonomous way of separating the concerns. After averaging more than 900 models from different industries they introduced a "(...) semi-automatic approach to identify decision logic in business processes (...)" \citep{BatoulisMeyerBazhenovaEtAl2015}. This semi-automatic approach incorporates the 900+ models they used to identify patterns in decision modeling. Formalization is not one of the key issues in this thesis, but the translation of BPMN to DMN or the link between them definitely is. 
Evaluating the compatibility of DMN with different modeling languages has been the objective of \citeauthor*{MertensGaillyPoels2015}. They approached a combined solution for knowledge-intensive work modeling and extracting the decision logic, what lead them to a new language called \textit{Declare-R-DMN}. Although the Declare language is not part of this thesis, the combination of it with DMN is useful to evaluate the compatibility with BPMN and CMMN. 

\section{Solution approach}
The research questions mentioned in \textit{Section 1} imply a thorough literature analysis. Both, the DMN and CMMN standard will be investigated from where the standard has been derived historically and how modeling before the standardization looked like. 
Afterwards each standard will be evaluated. Is the standardization good enough for generating the desired output? Are there enough meta elements to provide proper compatibility with BPMN and / or CMMN, DMN? This will be followed by examining the interfaces to other modeling languages. How flexible is the standard? Is there a way to create a whole new model with BPMN, DMN and CMMN according to the \textit{separation of concerns}? This will altogether be part of a literature analysis. 
The practical part of this thesis consists of the processes provided by the eKultur GmbH\footnote[1]{see http://www.ekulturportal.de}. eKultur GmbH develops a platform for touring theater companies where they can exchange master data from different participants, communicate with contracting parties and third parties or even with their own employees on tour. Development can be done in several ways with various strategies. In this case, eKultur GmbH decided to apply model-driven development as the domain experts and IT experts cannot work together on a daily basis. The whole project is assisted by the Federal Ministry for Economic Affairs and Energy and part of their research activity, which leads to several cooperations with other universities, such as the University of Regensburg, and associated contractors, e.g. Die Theaterinitiative. 
Model-driven development means creating a vast amount of models "supernetting" them to a large landscape of models. In this process, information will be stored in documents, models, sketches and other ways, but not always in a model. The pratical part of this thesis investigates the stored information and transforms it into models according to the CMMN and DMN standards. Afterwards, the modeling will be evaluated by the obtained insights of the literature analysis. The objective is to transform misused information from BPMN models and stored information from documents etc. into models that enhance the software engineering process and enable run-time changes of parameters. 

\section{Evaluation}
As mentioned in \textit{Section 3} this thesis provides a lot of literature research. The evaluation here can be done in two different ways. On the one hand, the literature analysis is supposed to result in a proper comparison between the standardization and requirements of case and decision modeling. On the other hand the research also manifests itself in the process modeling, which can be evaluated in terms of information handling and separating the concerns. In the end, this thesis is supposed to show the advantages and disadvantages of the two specifications CMMN and DMN, where they derived from and what can be improved in the future. Additionally, the compatibility with BPMN is shown and whether the specification improve the overall information processing and ability to change parameters in run-time. Camunda\footnote[2]{http://www.camunda.org} offers a way to measure information flow in BPMN models. For this thesis, they offer an expanded trial version of the camunda enterprise suite which lets me measure the BPMN information flow and visualize it in a heatmap. Yet this is not possible for DMN and CMMN, but in development. Depending on the release date a usage of these features for DMN and CMMN would be a good way to additionally evaluate the information flow. 

\bibliographystyle{chicago}
\bibliography{literatur/BA}

\end{document}